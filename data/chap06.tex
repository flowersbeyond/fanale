\chapter{总结与展望}
\label{cha:sum}

\section{工作总结}%1
实现错误自动修复一直是软件工程研究的美好愿景。近年来基于“生成-检验”框架的程序自动修复系统由于输入简单、错误类型限制小引起了很多研究者的注意。然而现有研究工作所实现的系统修复正确率和系统效率都难以令人满意,“生成-检验”系统推向实际应用仍有一定的距离。

基于上述背景,本文从系统内部模块优化、系统框架扩展两个方面提出提高系统的修复正确率和效率的技术方案。首先,针对框架内部“错误定位模块”,本文提出不完全正确“测试准则”存在这一事实,并以实验说明测试准则错误对“生成-检验”系统中常用的基于频谱的错误定位算法(SFL)错误定位精度的负面影响。针对这一问题,本文提出测试准则错误修复算法,使系统中错误定位算法尽量不受测试准则错误的负面影响。接着,本文针对系统的核心计算模块“搜索引擎”提出搜索优化算法。针对搜索引擎中与表达式替换、修改相关的修复模板,本文提出“预过滤”算法。该算法根据测试集中已通过和未通过的测试用例的运行时状态在备选修复方案进入到检验器前过滤掉其中不符合要求的修复方案,压缩搜索空间,减轻检验器的工作负担。实验表明,“预过滤”策略能够过滤掉搜索空间中约90\%的修复方案,系统效率有较大提升,在Defects4J上的实验效果也说明本文实现系统的修复能力已超过现有其他系统。

在模块优化工作基础上,本文提出在框架层扩展系统,进一步提高系统的实用性。首先,本文提出为使用者提供与系统交互的接口,使得系统能够利用使用者的经验减少无用搜索操作,提高系统效率。接着,本文提出系统应提供二次开发接口,使得开发人员能够根据需求方便的引入针对特定类型错误的修复算法,从而使系统本身的修复能力得到补充。

最后,本文将以上算法与扩展方案实现在原型系统SmartDebug中,并将该原型集成在Eclipse平台中供开发者在日常开发过程中使用。

\section{研究展望}

“生成-检验”系统的研究前景非常广阔。在本文工作基础上,有以下几个问题可以继续深入研究。

首先,本文提出的预过滤算法应用场景仍有一定限制。目前预过滤算法只能应用于与表达式替换、修改相关的修复模板生成的修复方案上。如何利用类似的思想将其扩展到一般修复方案中是一个有价值的研究问题。此外,目前算法实现中仍有许多基于经验假设的算法设计,例如判断两个表达式等价的标准、表达式等价与测试用例测试结果之间的关系假设等。如能在此处引入静态分析、逻辑推理等更加严格的方法,则算法的适用场景将更广泛。

其次,本文提出将使用者对程序错误的理解引入系统计算过程中,使得系统能够更准确地定位错误。在目前的设计中,使用者需要手动设置程序断点,引导系统运行程序。更友好的方式是系统能够根据计算结果提示用户应将断点设置在何处。这一提示算法也是值得探究的设计问题。

最后,本文提出了将系统模块接口开放,使系统可以方便引入针对特定类型错误的修复算法。本文仅以空指针错误为示例实现了NPEDebug系统。而更多其他类型错误的修复算法与“生成-检验”框架的结合应能使“生成-检验”系统的修复能力更进一步。