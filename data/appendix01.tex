\chapter{公式2-2和2-3的证明}
\label{cha:proof}

Let $\mathbf{P}(*)$ denote the probability of $*$. Let $O_c$ be the correct test oracle.
There are two cases of false negatives and two cases of false positives. For expression simplicity, we encode them as follows: \\
1. $P_O F_{O_c} P_{O'}$:$O(t_i) = \mathcal{P} \wedge O_c(t_i) = \mathcal{F} \wedge O'(t_i) = \mathcal{P}$ \\
2. $F_O P_{O_c} F_{O'}$:$O(t_i) = \mathcal{F} \wedge O_c(t_i) = \mathcal{P} \wedge O'(t_i) = \mathcal{F}$ \\
3. $P_O P_{O_c} F_{O'}$:$O(t_i) = \mathcal{P} \wedge O_c(t_i) = \mathcal{P} \wedge O'(t_i) = \mathcal{F}$ \\
4. $F_O F_{O_c} P_{O'}$:$O(t_i) = \mathcal{F} \wedge O_c(t_i) = \mathcal{F} \wedge O'(t_i) = \mathcal{P}$

Let $r$ be the error rate of the test oracle, i.e. the ratio of faulty oracle judgements to all oracle judgements. Then the probability that $t_i$ is false negative is
\begin{equation}
\label{equ:FN}
\begin{aligned}
& \mathbf{P}(fn(t_i))= \mathbf{P}(P_O F_{O_c} P_{O'}) + \mathbf{P}(F_O P_{O_c} F_{O'})
\end{aligned}
\end{equation}
And the probability that $t_i$ is false positive is
\begin{equation}
\begin{aligned}
\label{equ: FP}
& \mathbf{P}(fp(t_i))= \mathbf{P}(P_O F_{O_c} P_{O'}) + \mathbf{P}(F_O P_{O_c} F_{O'})
\end{aligned}
\end{equation}

And now we calculate $\mathbf{P}(P_O F_{O_c} P_{O'})$, $\mathbf{P}(F_O P_{O_c} F_{O'})$, $\mathbf{P}(P_O F_{O_c} P_{O'})$ and $\mathbf{P}(F_O P_{O_c} F_{O'})$.

\begin{equation}
\label{equ: P and F and P}
\begin{aligned}
& \mathbf{P}(P_O F_{O_c} P_{O'}) \\
= & \mathbf{P}(O(t_i) = \mathcal{P} \wedge O_c(t_i) = \mathcal{F} \wedge O'(t_i) = \mathcal{P}) \\
= & \mathbf{P}(O'(t_i) = \mathcal{P} | O(t_i) = \mathcal{P} \wedge O_c(t_i) = \mathcal{F}) \cdot \mathbf{P}(O(t_i) = \mathcal{P} \wedge O_c(t_i) = \mathcal{F})
\end{aligned}
\end{equation}

\begin{equation}
\label{equ: P | P and F}
\begin{aligned}
& \mathbf{P}(O'(t_i) = \mathcal{P} | O(t_i) = \mathcal{P} \wedge O_c(t_i) = \mathcal{F}) \\
= & \mathbf{P}(Suspicion(t_i) \le thres) \\
= & \mathbf{P}(\frac{Vote_{if}}{Vote_{ip} + Vote_{if}} \le thres) \\
= & \mathbf{P}(\frac{\sum_{t_{iq}\in T_{i,f}}^{}Sim(t_i, t_{iq})}{\sum_{t_{iq}\in T_i}^{}Sim(t_i, t_{iq})} \le thres)
\end{aligned}
\end{equation}

We assume that test cases in $T_i$ are similar enough to $t_i$ and the similarity are almost the same. Then approximately $\forall t_{iq} \in T_i$, $Sim(t_i, t_{iq}) \approx {sim}_i$, where $sim_i$ represents the average similarity of all $Sim(t_i, t_{iq})$. Then we have
\begin{equation}
\label{equ: sigma w}
\begin{aligned}
& \mathbf{P}(\frac{\sum_{t_{iq}\in T_{i,f}}^{}Sim(t_i, t_{iq})}{\sum_{t_{iq}\in T_i}^{}Sim(t_i, t_{iq})} \le thres) \\
\approx & \mathbf{P}(\frac{|T_{i,f}| \times {sim}_i}{|T_i| \times {sim}_i} \le thres) \\
= & \mathbf{P}(\frac{|T_{i,f}|}{|T_i|} \le thres) \\
= & \mathbf{P} (|T_{i,f}| \le |T_i| \times thres) \\
= & \sum_{w = 0}^{\hat{n}}{\mathbf{P}(|T_{i,f}| = w)}
\end{aligned}
\end{equation}
where $\hat{n} = \lfloor n \times thres \rfloor$.

Precise calculation of $\mathbf{P}(|T_{i,f}| = w)$ (given $w$) is obviously very complicated. Fortunately in our case an accurate estimation is good enough. To simplify the problem, we make the following two assumptions:

\textit{Assump. 1} : $\forall t_{iq}\in T_i \cup \{t_i\}$, $O_c(t_{iq}) \in \{\mathcal{P}, \mathcal{F}\}$  $i.i.d.$.

\textit{Assump. 2} : $\forall t_{iq_1}$, $t_{iq_2} \in T_i \cup \{t_i\}$, $iq_1 \ne iq_2$, $\mathbf{P}(O_c(t_{iq_1}) = O_c(t_{iq_2})) = {sim}_i\  (constant)$

The reasons for making such assumptions are explained in Section 3.3 thus not repeated here. Based on these two assumptions,
\begin{equation*}
\begin{aligned}
& \mathbf{P}(|T_{i,f}| = w) \\
= & \mathrm{C}_n^w{(\mathbf{P}(O(t_{iq}) = \mathcal{F}))}^w {(\mathbf{P}(O(t_{iq}) = \mathcal{P}))}^{(n-w)} \\
= & \mathrm{C}_n^w {(\mathbf{P}(O_c(t_{iq}) = \mathcal{F})\mathbf{P}(O(t_{iq}) = O_c(t_{iq}))+ \mathbf{P}(O_c(t_{iq}) = \mathcal{P})\mathbf{P}(O(t_{iq}) \neq O_c(t_{iq})))}^w \\
& \cdot {(\mathbf{P}(O_c(t_{iq}) = \mathcal{P})\mathbf{P}(O(t_{iq}) = O_c(t_{iq})) + \mathbf{P}(O_c(t_{iq}) = \mathcal{F})\mathbf{P}(O(t_{iq}) \neq O_c(t_{iq})))}^{(n - w)}
\end{aligned}
\end{equation*}
where $t_{iq}$ is an arbitrary test case in $T_i$.

Remember $r$ is the error rate of the test oracle, then $\mathbf{P}(O(t_{iq}) = O_c(t_{iq})) = 1 - r$, $\mathbf{P}(O(t_{iq}) \neq O_c(t_{iq})) = r$. Use $\beta_i$ to represent $\mathbf{P}(O_c(t_{iq}) = \mathcal{F}), t_{iq}\in T_i$, then $\mathbf{P}(O_c(t_{iq}) = \mathcal{P})= 1 - \beta_i$, and
\begin{equation}
\label{equ: single w}
\begin{aligned}
& \mathbf{P}(|T_{i,f}| = w) \\
= & \mathrm{C}_n^w{(\beta_i(1 - r) + (1-\beta_i)r)}^w{((1-\beta_i)(1 - r) + \beta_i r)}^{(n - w)} \\
= & \mathrm{C}_n^w{(\beta_i + r - 2\beta_i r)^w (1-(\beta_i + r - 2\beta_i r))^{(n-w)}}
\end{aligned}
\end{equation}

$r$ is completely dependent on the oracle error itself, so we cannot compute its value from elsewhere. However, $\beta_i$ can be deducted through ${sim}_i$.
$\forall t_{iq_1}$, $t_{iq_2} \in T_i \cup \{t_i\}$, $iq_1 \ne iq_2$, according to \textit{Assump. 1},
\begin{equation}
\begin{aligned}
& \mathbf{P}(O_c(t_{iq_1}) = O_c(t_{iq_2})) \\
= & \mathbf{P}(O_c(t_{iq_1}) = \mathcal{P}) \mathbf{P}(O_c(t_{iq_2}) = \mathcal{P}) + \mathbf{P}(O_c(t_{iq_1})=  \mathcal{F}) \mathbf{P}(O_c(t_{iq_2}) = \mathcal{F}) \\
= & (1-\beta_i)^2 + \beta_i^2
\end{aligned}
\end{equation}
while according to \textit{Assump. 2},
$$\mathbf{P}(O_c(t_{iq_1}) = O_c(t_{iq_2})) = {sim}_i$$
Therefore,
$$
(1-\beta_i)^2 + \beta_i^2 = {sim}_i
$$
Solving this equation, we get $\beta_{i1} = \frac{1 + \sqrt{2 {sim}_i - 1}}{2} or \beta_{i2} = \frac{1 - \sqrt{2 {sim}_i - 1}}{2}$.
Since we have assumed that all test cases in $T_i \cup \{t_i\}$ are similar enough, ${sim}_i$ should be close to 1, and $2 {sim}_i - 1 > 0$. As we know that $O_c(t_i) = \mathcal{F}$, therefore
\begin{equation}
\label{equ: beta}
\mathbf{P}(O_c(t_{iq}) = \mathcal{F}) = \beta_{i1} = \frac{1 + \sqrt{2 {sim}_i - 1}}{2}
\end{equation}
Synthesizing equation \ref{equ: P | P and F}, \ref{equ: sigma w} and \ref{equ: single w},
\begin{equation}
\label{equ: P | P and F synth}
\begin{aligned}
& \mathbf{P}(O'(t_i) = \mathcal{P} | O(t_i) = \mathcal{P} \wedge O_c(t_i) = \mathcal{F}) \\
= & \sum_{w = 0}^{\hat{n}}{\mathrm{C}_n^w{(\beta_{i1} + r - 2\beta_{i1} r)^w (1-(\beta_{i1} + r - 2\beta_{i1} r))^{(n-w)}}}
\end{aligned}
\end{equation}

Similarly, we have
\begin{equation}
\label{equ: F and P and F}
\begin{aligned}
& \mathbf{P}(F_O P_{O_c} F_{O'}) \\
= & \mathbf{P}(O(t_i) = \mathcal{F} \wedge O_c(t_i) = \mathcal{P} \wedge O'(t_i) = \mathcal{F}) \\
= & \mathbf{P}(O'(t_i) = \mathcal{F} | O(t_i) = \mathcal{F} \wedge O_c(t_i) = \mathcal{P}) \cdot \mathbf{P}(O(t_i) = \mathcal{F} \wedge O_c(t_i) = \mathcal{P})
\end{aligned}
\end{equation}
and
\begin{equation}
\begin{aligned}
& \mathbf{P}(O'(t_i) = \mathcal{F} | O(t_i) = \mathcal{F} \wedge O_c(t_i) = \mathcal{P}) \\
= & \sum_{w = 0}^{\hat{n}}{\mathbf{P}(|T_{i,p}| = w)} \\
= & \sum_{w = 0}^{\hat{n}}
{
	\mathrm{C}_n^w
	\cdot{(\mathbf{P}(O(t_{iq}) = \mathcal{P}))}^w 
	\cdot {(\mathbf{P}(O(t_{iq}) = \mathcal{F}))}^{(n-w)}
}
\\
= & \sum_{w = 0}^{\hat{n}}
{	\mathrm{C}_n^w
	\cdot {((1-\beta_i)(1-r) + \beta_i r)}^w
	\cdot {(\beta_i (1 - r) + (1- \beta_i)r)}^{(n - w)}
}
\end{aligned}
\end{equation}
where $\beta$ represents $\mathbf{P}(O_c(t_{iq}) = \mathcal{F})$. In this case, $O_c(t_i) = \mathcal{P}$, therefore
\begin{equation*}
\mathbf{P}(O_c(t_{iq}) = \mathcal{F}) = \beta_{i2} = \frac{1 - \sqrt{2 {sim}_i - 1}}{2}
\end{equation*}

Since $\beta_{i1} + \beta_{i2} = 1$, we have
\begin{equation}
\label{equ: F | F and P synth}
\begin{aligned}
& \mathbf{P}(O'(t_i) = \mathcal{F} | O(t_i) = \mathcal{F} \wedge O_c(t_i) = \mathcal{P}) \\
= & \sum_{w = 0}^{\hat{n}} {\mathrm{C}_n^w{((1-\beta_{i2})(1-r) + \beta_{i2} r)}^w} {{(\beta_{i2} (1 - r) + (1- \beta_{i2})r)}^{(n - w)}} \\
= & \sum_{w = 0}^{\hat{n}}\mathrm{C}_n^w {(\beta_{i1}(1-r) + (1 - \beta_{i1}) r)}^w 
				 { {((1 - \beta_{i1}) (1 - r) + \beta_{i1} r)}^{(n - w)}} \\
= & \mathbf{P}(O'(t_i) = \mathcal{P} | O(t_i) = \mathcal{P} \wedge O_c(t_i) = \mathcal{F})
\end{aligned}
\end{equation}
Synthesizing equations \ref{equ:FN}, \ref{equ: F and P and F}, \ref{equ: P and F and P}, \ref{equ: P | P and F synth} and \ref{equ: F | F and P synth}, we have
\begin{equation}
\begin{aligned}
& \mathbf{P}(fn(t_i))= \mathbf{P}(P_O F_{O_c} P_{O'}) + \mathbf{P}(F_O P_{O_c} F_{O'}) \\
= & \mathbf{P}(O(t_i) = \mathcal{P} \wedge O_c(t_i) = \mathcal{F} \wedge O'(t_i) = \mathcal{P}) + \mathbf{P}(O(t_i) = \mathcal{F} \wedge O_c(t_i) = \mathcal{P} \wedge O'(t_i) = \mathcal{F}) \\
= & \mathbf{P}(O'(t_i) = \mathcal{P} | O(t_i) = \mathcal{P} \wedge O_c(t_i) = \mathcal{F}) \cdot \mathbf{P}(O(t_i) = \mathcal{P} \wedge O_c(t_i) = \mathcal{F}) \\
& + \mathbf{P}(O'(t_i) = \mathcal{F} | O(t_i) = \mathcal{F} \wedge O_c(t_i) = \mathcal{P}) \cdot \mathbf{P}(O(t_i) = \mathcal{F} \wedge O_c(t_i) = \mathcal{P}) \\
= & \mathbf{P}(O'(t_i) = \mathcal{P} | O(t_i) = \mathcal{P} \wedge O_c(t_i) = \mathcal{F}) \\
&\cdot {\mathbf{P}(O(t_i) = \mathcal{P} \wedge O_c(t_i) = \mathcal{F})} { +\mathbf{P}(O(t_i) = \mathcal{F} \wedge O_c(t_i) = \mathcal{P})} \\
= & \mathbf{P}(O'(t_i) = \mathcal{P} | O(t_i) = \mathcal{P} \wedge O_c(t_i) = \mathcal{F}) \cdot \mathbf{P}(O(t_i) \ne O_c(t_i)) \\
= & \mathbf{P}(O'(t_i) = \mathcal{P} | O(t_i) = \mathcal{P} \wedge O_c(t_i) = \mathcal{F}) \cdot r \\
\approx & r \sum_{w = 0}^{\hat{n}}{\mathrm{C}_n^w{\phi_i^w (1-\phi_i)^{(n-w)}}}
\end{aligned}
\end{equation}
where $\phi_i = \beta_{i1} + r - 2\beta_{i1} r$.

Following the same routine,
\begin{equation}
\begin{aligned}
& \mathbf{P}(P_O P_{O_c} F_{O'}) \\
= & \mathbf{P}(O(t_i) = \mathcal{P} \wedge O_c(t_i) = \mathcal{P} \wedge O'(t_i) = \mathcal{F}) \\
= & \mathbf{P}(O'(t_i) = \mathcal{F} | O(t_i) = \mathcal{P} \wedge O_c(t_i) = \mathcal{P}) \cdot \mathbf{P}(O(t_i) = \mathcal{P} \wedge O_c(t_i) = \mathcal{P}) \\  
\end{aligned}
\end{equation}
and
\begin{equation}
\begin{aligned}
& \mathbf{P}(O'(t_i) = \mathcal{F} | O(t_i) = \mathcal{P} \wedge O_c(t_i) = \mathcal{P}) \\
\approx & \sum_{w = \hat{n} + 1}^{n}{\mathbf{P}(|T_{i,f}| = w)} \\
= & \sum_{w = \hat{n} + 1}^{n}
{  \mathrm{C}_n^w{(1 - (\beta_{i1} + r - 2\beta_{i1} r))^w}  }
{  (\beta_{i1} + r - 2\beta_{i1} r)^{(n-w)}  } \\
= & \mathbf{P}(O'(t_i) = \mathcal{P} | O(t_i) = \mathcal{F} \wedge O_c(t_i) = \mathcal{F})
\end{aligned}
\end{equation}
then
\begin{equation}
\begin{aligned}
& \mathbf{P}(fp(t_i))= \mathbf{P}(P_O P_{O_c} F_{O'}) + \mathbf{P}(F_O F_{O_c} P_{O'}) \\
= & (\mathbf{P}(O(t_i) = \mathcal{P} \wedge O_c(t_i) = \mathcal{P} \wedge O'(t_i) = \mathcal{F})) 
 + (\mathbf{P}(O(t_i) = \mathcal{F} \wedge O_c(t_i) = \mathcal{F} \wedge O'(t_i) = \mathcal{P})) \\
= & \mathbf{P}(O'(t_i) = \mathcal{F} | O(t_i) = \mathcal{P} \wedge O_c(t_i) = \mathcal{P}) \cdot \mathbf{P}(O(t_i) = \mathcal{P} \wedge O_c(t_i) = \mathcal{P}) \\
& + \mathbf{P}(O'(t_i) = \mathcal{P} | O(t_i) = \mathcal{F} \wedge O_c(t_i) = \mathcal{F}) \cdot \mathbf{P}(O(t_i) = \mathcal{F} \wedge O_c(t_i) = \mathcal{F}) \\
= & \mathbf{P}(O'(t_i) = \mathcal{F} | O(t_i) = \mathcal{P} \wedge O_c(t_i) = \mathcal{P}) \\
& \cdot{\mathbf{P}(O(t_i) = \mathcal{P} \wedge O_c(t_i) = \mathcal{P})}{ +\mathbf{P}(O(t_i) = \mathcal{F} \wedge O_c(t_i) = \mathcal{F})} \\
= & \mathbf{P}(O'(t_i) = \mathcal{P} | O(t_i) = \mathcal{P} \wedge O_c(t_i) = \mathcal{F}) \cdot \mathbf{P}(O(t_i) = O_c(t_i)) \\
= & \mathbf{P}(O'(t_i) = \mathcal{P} | O(t_i) = \mathcal{P} \wedge O_c(t_i) = \mathcal{F}) \cdot (1-r) \\
\approx & (1-r)
\sum_{w = \hat{n} + 1}^{n} \mathrm{C}_n^w{(1 - \phi_i)^w} \phi_i^{(n-w)}
\end{aligned}
\end{equation}
where $\phi_i = \beta_{i1} + r - 2 \beta_{i1} r$.

Finally, we get
\begin{equation}
\mathbf{P}(fn(t_i)) \approx r \sum_{w = 0}^{\hat{n}}{\mathrm{C}_n^w{\phi_i^w (1-\phi_i)^{(n-w)}}} \\
\end{equation}
\begin{equation}
\mathbf{P}(fp(t_i)) \approx (1 - r) \sum_{w = \hat{n} + 1}^{n} \mathrm{C}_n^w{(1 - \phi_i)^w} \phi_i^{(n-w)}
\end{equation}

where $\phi_i = \beta_{i1} + r - 2 \beta_{i1} r$, $\beta_{i1} = \frac{1 + \sqrt{2 {sim}_i - 1}}{2}$,
${sim}_i$ is the average similarity of all test cases in the same $T_i$, $\hat{n} = \lfloor n \times thres \rfloor$, $r$ is the error rate of the test oracle.