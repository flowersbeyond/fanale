\chapter{“生成-检验”系统的扩展}
\label{cha:ext}

\section{引言}%2
\section{相关工作}%2

\section{交互式调试}%8
介绍“交互式调试”的基本思想,阐述将其引入后“生成-检验”框架应做出的相应调整及其对系统错误定位模块和搜索模块的正面作用。
\section{针对单类别错误的可扩展框架}%10
举例说明“生成-检验”系统在修复特定类别错误上的局限性,阐述将针对特定类别错误修复算法整合进“生成-检验”系统中的架构设计,以空指针(NPE)为例具体说明该架构的可扩展性。
\subsection{框架设计}%3
\subsection{扩展示例}%7
单类别错误修复实例:在CWE-Null-Dereference测试集上的评测结果。
\section{本章小结}%1